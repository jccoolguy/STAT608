% Options for packages loaded elsewhere
\PassOptionsToPackage{unicode}{hyperref}
\PassOptionsToPackage{hyphens}{url}
\PassOptionsToPackage{dvipsnames,svgnames,x11names}{xcolor}
%
\documentclass[
  letterpaper,
  DIV=11,
  numbers=noendperiod]{scrartcl}

\usepackage{amsmath,amssymb}
\usepackage{lmodern}
\usepackage{iftex}
\ifPDFTeX
  \usepackage[T1]{fontenc}
  \usepackage[utf8]{inputenc}
  \usepackage{textcomp} % provide euro and other symbols
\else % if luatex or xetex
  \usepackage{unicode-math}
  \defaultfontfeatures{Scale=MatchLowercase}
  \defaultfontfeatures[\rmfamily]{Ligatures=TeX,Scale=1}
\fi
% Use upquote if available, for straight quotes in verbatim environments
\IfFileExists{upquote.sty}{\usepackage{upquote}}{}
\IfFileExists{microtype.sty}{% use microtype if available
  \usepackage[]{microtype}
  \UseMicrotypeSet[protrusion]{basicmath} % disable protrusion for tt fonts
}{}
\makeatletter
\@ifundefined{KOMAClassName}{% if non-KOMA class
  \IfFileExists{parskip.sty}{%
    \usepackage{parskip}
  }{% else
    \setlength{\parindent}{0pt}
    \setlength{\parskip}{6pt plus 2pt minus 1pt}}
}{% if KOMA class
  \KOMAoptions{parskip=half}}
\makeatother
\usepackage{xcolor}
\setlength{\emergencystretch}{3em} % prevent overfull lines
\setcounter{secnumdepth}{-\maxdimen} % remove section numbering
% Make \paragraph and \subparagraph free-standing
\ifx\paragraph\undefined\else
  \let\oldparagraph\paragraph
  \renewcommand{\paragraph}[1]{\oldparagraph{#1}\mbox{}}
\fi
\ifx\subparagraph\undefined\else
  \let\oldsubparagraph\subparagraph
  \renewcommand{\subparagraph}[1]{\oldsubparagraph{#1}\mbox{}}
\fi

\usepackage{color}
\usepackage{fancyvrb}
\newcommand{\VerbBar}{|}
\newcommand{\VERB}{\Verb[commandchars=\\\{\}]}
\DefineVerbatimEnvironment{Highlighting}{Verbatim}{commandchars=\\\{\}}
% Add ',fontsize=\small' for more characters per line
\usepackage{framed}
\definecolor{shadecolor}{RGB}{241,243,245}
\newenvironment{Shaded}{\begin{snugshade}}{\end{snugshade}}
\newcommand{\AlertTok}[1]{\textcolor[rgb]{0.68,0.00,0.00}{#1}}
\newcommand{\AnnotationTok}[1]{\textcolor[rgb]{0.37,0.37,0.37}{#1}}
\newcommand{\AttributeTok}[1]{\textcolor[rgb]{0.40,0.45,0.13}{#1}}
\newcommand{\BaseNTok}[1]{\textcolor[rgb]{0.68,0.00,0.00}{#1}}
\newcommand{\BuiltInTok}[1]{\textcolor[rgb]{0.00,0.23,0.31}{#1}}
\newcommand{\CharTok}[1]{\textcolor[rgb]{0.13,0.47,0.30}{#1}}
\newcommand{\CommentTok}[1]{\textcolor[rgb]{0.37,0.37,0.37}{#1}}
\newcommand{\CommentVarTok}[1]{\textcolor[rgb]{0.37,0.37,0.37}{\textit{#1}}}
\newcommand{\ConstantTok}[1]{\textcolor[rgb]{0.56,0.35,0.01}{#1}}
\newcommand{\ControlFlowTok}[1]{\textcolor[rgb]{0.00,0.23,0.31}{#1}}
\newcommand{\DataTypeTok}[1]{\textcolor[rgb]{0.68,0.00,0.00}{#1}}
\newcommand{\DecValTok}[1]{\textcolor[rgb]{0.68,0.00,0.00}{#1}}
\newcommand{\DocumentationTok}[1]{\textcolor[rgb]{0.37,0.37,0.37}{\textit{#1}}}
\newcommand{\ErrorTok}[1]{\textcolor[rgb]{0.68,0.00,0.00}{#1}}
\newcommand{\ExtensionTok}[1]{\textcolor[rgb]{0.00,0.23,0.31}{#1}}
\newcommand{\FloatTok}[1]{\textcolor[rgb]{0.68,0.00,0.00}{#1}}
\newcommand{\FunctionTok}[1]{\textcolor[rgb]{0.28,0.35,0.67}{#1}}
\newcommand{\ImportTok}[1]{\textcolor[rgb]{0.00,0.46,0.62}{#1}}
\newcommand{\InformationTok}[1]{\textcolor[rgb]{0.37,0.37,0.37}{#1}}
\newcommand{\KeywordTok}[1]{\textcolor[rgb]{0.00,0.23,0.31}{#1}}
\newcommand{\NormalTok}[1]{\textcolor[rgb]{0.00,0.23,0.31}{#1}}
\newcommand{\OperatorTok}[1]{\textcolor[rgb]{0.37,0.37,0.37}{#1}}
\newcommand{\OtherTok}[1]{\textcolor[rgb]{0.00,0.23,0.31}{#1}}
\newcommand{\PreprocessorTok}[1]{\textcolor[rgb]{0.68,0.00,0.00}{#1}}
\newcommand{\RegionMarkerTok}[1]{\textcolor[rgb]{0.00,0.23,0.31}{#1}}
\newcommand{\SpecialCharTok}[1]{\textcolor[rgb]{0.37,0.37,0.37}{#1}}
\newcommand{\SpecialStringTok}[1]{\textcolor[rgb]{0.13,0.47,0.30}{#1}}
\newcommand{\StringTok}[1]{\textcolor[rgb]{0.13,0.47,0.30}{#1}}
\newcommand{\VariableTok}[1]{\textcolor[rgb]{0.07,0.07,0.07}{#1}}
\newcommand{\VerbatimStringTok}[1]{\textcolor[rgb]{0.13,0.47,0.30}{#1}}
\newcommand{\WarningTok}[1]{\textcolor[rgb]{0.37,0.37,0.37}{\textit{#1}}}

\providecommand{\tightlist}{%
  \setlength{\itemsep}{0pt}\setlength{\parskip}{0pt}}\usepackage{longtable,booktabs,array}
\usepackage{calc} % for calculating minipage widths
% Correct order of tables after \paragraph or \subparagraph
\usepackage{etoolbox}
\makeatletter
\patchcmd\longtable{\par}{\if@noskipsec\mbox{}\fi\par}{}{}
\makeatother
% Allow footnotes in longtable head/foot
\IfFileExists{footnotehyper.sty}{\usepackage{footnotehyper}}{\usepackage{footnote}}
\makesavenoteenv{longtable}
\usepackage{graphicx}
\makeatletter
\def\maxwidth{\ifdim\Gin@nat@width>\linewidth\linewidth\else\Gin@nat@width\fi}
\def\maxheight{\ifdim\Gin@nat@height>\textheight\textheight\else\Gin@nat@height\fi}
\makeatother
% Scale images if necessary, so that they will not overflow the page
% margins by default, and it is still possible to overwrite the defaults
% using explicit options in \includegraphics[width, height, ...]{}
\setkeys{Gin}{width=\maxwidth,height=\maxheight,keepaspectratio}
% Set default figure placement to htbp
\makeatletter
\def\fps@figure{htbp}
\makeatother

\usepackage{amsmath}
\KOMAoption{captions}{tableheading}
\makeatletter
\makeatother
\makeatletter
\makeatother
\makeatletter
\@ifpackageloaded{caption}{}{\usepackage{caption}}
\AtBeginDocument{%
\ifdefined\contentsname
  \renewcommand*\contentsname{Table of contents}
\else
  \newcommand\contentsname{Table of contents}
\fi
\ifdefined\listfigurename
  \renewcommand*\listfigurename{List of Figures}
\else
  \newcommand\listfigurename{List of Figures}
\fi
\ifdefined\listtablename
  \renewcommand*\listtablename{List of Tables}
\else
  \newcommand\listtablename{List of Tables}
\fi
\ifdefined\figurename
  \renewcommand*\figurename{Figure}
\else
  \newcommand\figurename{Figure}
\fi
\ifdefined\tablename
  \renewcommand*\tablename{Table}
\else
  \newcommand\tablename{Table}
\fi
}
\@ifpackageloaded{float}{}{\usepackage{float}}
\floatstyle{ruled}
\@ifundefined{c@chapter}{\newfloat{codelisting}{h}{lop}}{\newfloat{codelisting}{h}{lop}[chapter]}
\floatname{codelisting}{Listing}
\newcommand*\listoflistings{\listof{codelisting}{List of Listings}}
\makeatother
\makeatletter
\@ifpackageloaded{caption}{}{\usepackage{caption}}
\@ifpackageloaded{subcaption}{}{\usepackage{subcaption}}
\makeatother
\makeatletter
\@ifpackageloaded{tcolorbox}{}{\usepackage[many]{tcolorbox}}
\makeatother
\makeatletter
\@ifundefined{shadecolor}{\definecolor{shadecolor}{rgb}{.97, .97, .97}}
\makeatother
\makeatletter
\makeatother
\ifLuaTeX
  \usepackage{selnolig}  % disable illegal ligatures
\fi
\IfFileExists{bookmark.sty}{\usepackage{bookmark}}{\usepackage{hyperref}}
\IfFileExists{xurl.sty}{\usepackage{xurl}}{} % add URL line breaks if available
\urlstyle{same} % disable monospaced font for URLs
\hypersetup{
  pdftitle={STAT 639 HW 4},
  pdfauthor={Jack Cunningham (jgavc@tamu.edu)},
  colorlinks=true,
  linkcolor={blue},
  filecolor={Maroon},
  citecolor={Blue},
  urlcolor={Blue},
  pdfcreator={LaTeX via pandoc}}

\title{STAT 639 HW 4}
\author{Jack Cunningham (jgavc@tamu.edu)}
\date{8/20/24}

\begin{document}
\maketitle
\ifdefined\Shaded\renewenvironment{Shaded}{\begin{tcolorbox}[frame hidden, enhanced, breakable, interior hidden, boxrule=0pt, borderline west={3pt}{0pt}{shadecolor}, sharp corners]}{\end{tcolorbox}}\fi

\[
\begin{bmatrix}
1&2&3\\
a&b&c
\end{bmatrix}
\]

Matrix Algebra Review

\(A = \begin{bmatrix} 1&0&2&3\\-1&2&0&-2 \end{bmatrix}\),
\(B=\begin{bmatrix} 0&-1\\3&0\\2&1\\0&-2 \end{bmatrix}\) ,
\(C=\begin{bmatrix} 1&0\\0&1\\0&0\end{bmatrix}\)

1.

\(A'= \begin{bmatrix} 1 &-1\\0&2\\2&0\\3&-2\end{bmatrix}\)

2.

\(A'+B=\begin{bmatrix} 1&-2\\3&2\\4&1\\3&-4\end{bmatrix}\)

3.

\(AB=\begin{bmatrix} 4 &-5\\6&5 \end{bmatrix}\)

4.

\(BA=\begin{bmatrix} 1&-2&0&2\\3&0&6&9\\1&2&4&4\\2&-4&0&4\end{bmatrix}\)

\(AB\neq BA\)

5.

\(AB\) is not singular. The matrix is invertible due to the fact that
the determinant is not zero.

\(\text{det}(AB)=|AB|=(4)(5)-(-5)(6)=50\)

6.

The trace is the sum of diagonal elements. \(\text{Tr}(AB) = 4 + 5=9\).

7.

\((AB)'=B'A'\)

8.

\((AB)^{-1}=\frac{1}{50} \begin{bmatrix} 5&5 \\ -6 & 4 \end{bmatrix}\)

9.

\(I_2 = \begin{bmatrix} 1 & 0 \\ 0 & 1 \end{bmatrix}\)

10.

\(I_2 A=A\). The identity matrix multiplied by a matrix does not change
the matrix.

11.

The column space of C is a plane on XY.

12.

The projection matrix for C is:

\(\begin{bmatrix} 1 & 0 & 0 \\ 0 & 1 & 0 \\ 0 & 0 & 0 \end{bmatrix}\)

13.

\(\begin{bmatrix} 1 & 0 & 0 \\ 0 & 1 & 0 \\ 0 & 0 & 0 \end{bmatrix} \begin{bmatrix} 2 \\ 2 \\ 2 \end{bmatrix}=\begin{bmatrix} 2 \\ 2 \\0 \end{bmatrix}\)

14.

The vector d pointing at \(\begin{bmatrix} 2 \\ 2 \\ 2 \end{bmatrix}\)
has been projected down to the XY plane losing its Z axis direction. It
now is a line with components
\(\begin{bmatrix} 2 \\ 2 \\0 \end{bmatrix}\).

15.

No vector d and f are not orthogonal. The dot product \(d'f=2\). In
order for two vectors to be orthogonal the dot product must equal zero.

16.

The dot product of \(1 \cdot 1\) where
\(1 = \begin{bmatrix} 1 & 1 & \cdot & \cdot & 1\end{bmatrix}\) with
length n is:

\(1 \cdot 1=1_11_1+1_21_2+...1_n1_n=\sum_{i=1}^n1_i1_i=\sum_{i=1}^n1_i=n\)

17.

The dot product of
\(1 \cdot X=1_1X_1+1_2X_2+...+1_nX_n=\sum_{i=1}^nX_i\) .

18.

The dot product of
\(X \cdot X=X_1X_1+X_2+...+X_nX_n=\sum_{i=1}^nX_i^2\).

19.

The first eigenvector is multiplied by \(\lambda\) when multiplied by A.
We stay on the same eigenvector space changing direction/magnitude each
time it is transformed by A.

II. Calculus Review

1.

\(6x +2y^2\)

2.

\(4xy -1\)

III. Log Review

1.

\(\text{log}(e)=1\)

2.

\(\text{log}(\frac{x}{y})=\text{log}(x)-\text{log}(y)\)

3.

\(\text{log}(x^n)=n\text{log}(x)\)

4.

\(\text{log}(x)=y,x=e^y\)

IV. Statistics and Linear Regression Review

1.

\(\hat{y}=1.1667x-129.1667\)

2.

\begin{Shaded}
\begin{Highlighting}[]
\FloatTok{1.1667}\SpecialCharTok{*}\DecValTok{160} \SpecialCharTok{{-}} \FloatTok{129.1667}
\end{Highlighting}
\end{Shaded}

\begin{verbatim}
[1] 57.5053
\end{verbatim}

3.

As height increases by 1 cm the prediction of weight increases by 1.1667
kg.

4.

The standard error of the slope is the measure of spread in the
distribution of the height parameter estimate.

5.

\(t_{n-2}=\frac{\hat{\beta_1}-\beta_1}{se(\hat{\beta})}\)

\begin{Shaded}
\begin{Highlighting}[]
\NormalTok{t\_statistic }\OtherTok{=}\NormalTok{ (}\FloatTok{1.1667} \SpecialCharTok{{-}} \DecValTok{0}\NormalTok{)}\SpecialCharTok{/}\FloatTok{0.1521}
\NormalTok{t\_statistic}
\end{Highlighting}
\end{Shaded}

\begin{verbatim}
[1] 7.670611
\end{verbatim}

6.

Yes, height and weight are linearly associated. The height parameter
estimate is statically significantly different than zero.

7.

There is evidence as height increases weight tends to increase.

8.

\begin{Shaded}
\begin{Highlighting}[]
\NormalTok{t\_alpha }\OtherTok{\textless{}{-}} \FunctionTok{qt}\NormalTok{(}\DecValTok{1} \SpecialCharTok{{-}} \FloatTok{0.05}\NormalTok{, }\DecValTok{8} \SpecialCharTok{{-}} \DecValTok{2}\NormalTok{)}
\NormalTok{lower }\OtherTok{\textless{}{-}} \FloatTok{1.1667} \SpecialCharTok{{-}}\NormalTok{ (}\FloatTok{0.1521}\SpecialCharTok{/}\FunctionTok{sqrt}\NormalTok{(}\DecValTok{8}\NormalTok{))}\SpecialCharTok{*}\NormalTok{t\_alpha}
\NormalTok{upper }\OtherTok{\textless{}{-}} \FloatTok{1.1667} \SpecialCharTok{+}\NormalTok{ (}\FloatTok{0.1521}\SpecialCharTok{/}\FunctionTok{sqrt}\NormalTok{(}\DecValTok{8}\NormalTok{))}\SpecialCharTok{*}\NormalTok{t\_alpha}
\FunctionTok{c}\NormalTok{(lower, upper)}
\end{Highlighting}
\end{Shaded}

\begin{verbatim}
[1] 1.062205 1.271195
\end{verbatim}

9.

If we were to repeat this experiment again we are 95\% confident that
the estimate of the height parameter would rest between 0.86878 and
1.46462.



\end{document}
